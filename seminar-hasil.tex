\documentclass{lib/skripsi}

\usepackage{blindtext, caption, csvsimple, fancybox, longtable, pdflscape, tocloft}

% Redefine caption format for regular tables
\captionsetup[table]{position=top,justification=raggedright}

% Redefine caption format for longtables
\captionsetup[longtable]{position=top, justification=raggedright}

% \pagestyle{fancy}
% \rhead{Tugas Akhir Program Studi Sistem Informasi}

%===========================================================
% Definisi Data Peneliti, Judul, Pembimbing dan Penguji
%-----------------------------------------------------------
\titleskripsi{PENGEMBANGAN SISTEM MONITORING MESIN DIESEL BERBASIS IOT PADA PT BISMA JAYA MENGGUNAKAN METODE EXTREME PROGRAMMING (XP)}

\fullname{Haidar Dzaky Sumpena}
\idnum{10201043}

\yearsubmit{2024}
\program{Sistem Informasi}
\faculty{Matematika dan Teknologi Informasi}
\university{Institut Teknologi Kalimantan}
\city{Balikpapan}

\firstsupervisor{Aidil Saputra Kirsan, S.Kom., M.Tr.Kom}
\firstsupervisorNIP{100320240}
\secondsupervisor{Henokh Lugo Hariyanto, M.Sc.}
\secondsupervisorNIP{199303062022041001}
% \firstexaminer{ - }
% \secondexaminer{ - }
%-----------------------------------------------------------
% End Definisi Data Peneliti, Judul, Pembimbing dan Penguji
%===========================================================

\begin{document}
    \noindent
    \coverproposal
    \pagenumbering{roman}

    %===========================================================
    % Kata pengantar, abstract
    %-----------------------------------------------------------

    \supervisorapprovalpage

    \addcontentsline{toc}{chapter}{KATA PENGANTAR}
    \chapter*{KATA PENGANTAR}

Puji syukur kepada Allah SWT atas berkat dan rahmat-Nya penulis dapat menyelesaikan laporan tugas akhir yang berjudul:

\begin{center}
    "PENGEMBANGAN SISTEM MONITORING MESIN DIESEL BERBASIS IOT PADA PT BISMA JAYA MENGGUNAKAN METODE EXTREME PROGRAMMING (XP)"
\end{center}

Laporan tugas akhir ini merupakan salah satu syarat yang harus
ditempuh untuk menyelesaikan Program Sarjana di Program Studi Sistem
Informasi, Jurusan Matematika dan Teknologi Informasi, Institut Teknologi
Kalimantan (ITK) Balikpapan. Besar terima kasih penulis ucapkan kepada:

\begin{enumerate}[topsep=0pt,itemsep=0pt,partopsep=0pt, parsep=0pt]

    \item Bapak Aidil Kirsan Saputra, S.Kom., M.Tr.Kom, selaku Dosen Pembimbing Utama dan
    Bapak Henokh Lugo Hariyanto, S.Si., M.Sc., selaku Dosen Pembimbing
    Pendamping.

    \item Ibu Sri Rahayu Natasia, S.Komp, M.Si., M.Sc., selaku Koordinator Program
    Studi Sistem Informasi Jurusan Matematika dan Teknologi Informasi ITK.

    \item Bapak/Ibu Dosen dan Bapak/Ibu Tendik Program Studi Sistem Informasi
    Jurusan Matematika dan Teknologi Informasi ITK.

    \item Serta semua pihak yang terlibat dalam penyusunan proposal tugas akhir ini.
\end{enumerate}

Penulis menyadari bahwa laporan tugas akhir ini masih jauh dari sempurna dikarenakan terbatasnya pengalaman dan pengetahuan yang dimiliki penulis. Oleh karena itu, penulis mengharapkan segala bentuk saran serta masukan bahkan kritik yang membangun dari berbagai pihak. Semoga tulisan ini memberikan manfaat kepada semua pihak yang membutuhkan dan terutama untuk penulis.

\vspace{0.5cm}
\begin{flushright}
    Balikpapan, 16 Januari 2024\\
    \vspace{2cm}
    {Haidar Dzaky Sumpena}\\
    NIM {10201043}
\end{flushright}


    \pagebreak

    \addcontentsline{toc}{chapter}{ABSTRAK}
    \chapter*{PENGEMBANGAN SISTEM MONITORING MESIN DIESEL BERBASIS IOT PADA PT BISMA JAYA MENGGUNAKAN METODE EXTREME PROGRAMMING (XP)}
\begin{table}[h]
    \begin{tabular}
        {
            p{.4\textwidth}
            p{.6\textwidth}
            }
            \\
            Nama & : Haidar Dzaky Sumpena
            \\
            NIM & : 10201043
            \\
            Dosen Pembimbing Utama &
            : Aidil Kirsan Saputra, S.Kom., M.Tr.Kom
            \\
            Dosen Pembimbing Pendamping &
            : Henokh Lugo Hariyanto, M.Sc.
        \end{tabular}
    \end{table}

\begin{center}
    \textbf{ABSTRAK}
\end{center}

Internet of Things merupakan teknologi yang dapat merevolusi industri melalui kontrol dan pemantauan secara jarak jauh. Teknologi ini dapat diterapkan di berbagai industri termasuk pada transportasi.Tantangan di industri ini adalah memastikan jumlah bahan bakar yang dilaporkan sesuai dengan nilai aktual yang dihabiskan. Sistem monitoring berbasis IoT yang akan dikembangkan dalam penelitian ini berfungsi sebagai penghubung mitra dengan armada yang beroperasi. Metode pengembangan perangkat lunak yang digunakan adalah Extreme Programming yang menekankan kolaborasi serta pengembangan yang dinamis. Diharapkan, dengan diterapkannya sistem monitoring berbasis IoT ini, mitra dapat melakukan pemantauan bahan bakar serta menjaga efektivitas armada yang beroperasi secara real time.


\begin{table}[h]
    \begin{tabular}{ p{0.17\textwidth} p{0.8\textwidth} }
        \\
        \textbf{Kata Kunci :} & \textit{Internet of Things}, sistem \textit{monitoring}, bahan bakar, efisiensi, \textit{extreme programming}
    \end{tabular}
\end{table}

    \pagebreak

    %===========================================================
    % Daftar isi, daftar gambar, daftar tabel
    %-----------------------------------------------------------
    \addcontentsline{toc}{chapter}{DAFTAR ISI}
    \tableofcontents
    \pagebreak

    \addcontentsline{toc}{chapter}{DAFTAR TABEL}
    \listoftables
    \pagebreak

    \addcontentsline{toc}{chapter}{DAFTAR GAMBAR}
    \listoffigures
    \pagebreak

    \addcontentsline{toc}{chapter}{DAFTAR LAMPIRAN}
    \listofappendices
    \pagebreak

    \pagenumbering{arabic}
    %-----------------------------------------------------------
    % End Daftar isi, daftar gambar, daftar tabel
    %===========================================================

    %===========================================================
    % Daftar masukan untuk Bab
    %-----------------------------------------------------------
    \chapter{PENDAHULUAN}

\section{Latar Belakang}

Indonesia merupakan negara maritim dengan luas laut dan perairan 62\% \parencite{inproc:wuryadani}. Potensi ini harus didukung berdasarkan prinsip \textit{Blue Economy}. \textit{Blue Economy} sendiri merupakan komponen penting dalam pengembangan keberlanjutan yang berfokus pada ekonomi maritim yang meliputi berbagai sektor seperti perikanan, akuakultur, dan transportasi maritim. Konsep \textit{Blue Economy} berkaitan erat dengan \textit{Sustainable Development Goal} ke 14 yang membahas mengenai pelestarian dan penggunaan lautan, laut, dan sumber daya laut secara berkelanjutan \parencite{misc:lse}. Menurut \textcite{misc:dephub}, transportasi laut memegang peran strategis untuk mendorong pertumbuhan ekonomi di Indonesia. Salah satu bentuk untuk mendukung rencana jangka panjang ini adalah melalui upaya menuntaskan permasalahan fundamental dari industri tersebut. Pada industri maritim, salah satu biaya dengan rasio komposisi terbesar terletak pada biaya bahan bakar operasional. Tingginya biaya bahan bakar ini dapat memperlambat kemajuan industri maritim dikarenakan akan mengurangi pendapatan perusahaan, terlebih jika ternyata tingginya biaya bahan bakar ini disebabkan oleh hal lain diluar operasional. Sehingga, memastikan bahwa penggunaan bakar lebih efisien dirasa perlu.

Efisiensi sendiri terbagi menjadi dua, yakni Efisiensi Teknologi dan Efisiensi Manajemen. Efisiensi Teknologi merujuk pada keterbaruan teknologi mesin yang mampu menghemat penggunaan bahan bakar dari waktu ke waktu. Sedangkan pada Efisiensi Manajemen memastikan bahwa bahan bakar sepenuhnya digunakan untuk mendukung operasi. Fokusan pada penelitian ini adalah Efisiensi Manajemen. Untuk itu, diperlukan sebuah teknologi untuk melakukan validasi data penggunaan bahan bakar yang dilaporkan dengan nilai aktual yang dihabiskan.

Teknologi transformatif tersebut bernama IoT atau \textit{Internet of Things} yang berpotensi merevolusi berbagai industri melalui kontrol dan pemantauan ekstensif secara jarak jauh \parencite{article:hercog}. Kemampuan teknologi IoT dalam memberikan data secara jarak jauh membuka jalan bagi pelaku industri untuk merealisasi efisiensi bahan bakar khususnya pada transportasi laut. Hal ini senada dengan penelitian yang dilakukan \textcite{article:suciu}, yang menyatakan IoT memungkinkan integrasi mesin, sistem, dan proses untuk meningkatkan efisiensi operasional dan \textit{predictive maintenance}.

Langkah untuk melakukan efisiensi dengan kontrol melalui teknologi IoT juga dinilai tepat mengingat minyak fosil akan habis di tahun 2070 \parencite{misc:bp} sehingga pelaku industri tidak hanya menghemat biaya operasional, tetapi secara tidak langsung juga menjaga lingkungan secara berkelanjutan. Dengan demikian, operasi akan dipastikan berjalan secara optimal dengan memanfaatkan bahan bakar secara maksimal. Sebaliknya, salah satu dampak yang ditimbulkan dari belum diterapkannya teknologi ini adalah kurangnya kontrol yang mengakibatkan celah pada pelanggaran hukum. Pada tahun 2020 terdapat kasus penggelapan bahan bakar yang mencapai 2.5 ton liter \parencite{misc:aditya}, sehingga menimbulkan kerugian negara mencapai 710 juta rupiah. Hal ini dapat diatasi menggunakan sistem monitoring berbasis IoT yang memungkinkan pemantauan secara jarak jauh. Sistem monitoring berbasis IoT yang dimaksud adalah suatu sistem yang menggunakan \textit{Internet of Things} untuk memantau dan menyimpan data dari berbagai sensor. Dalam penelitian ini maka akan dikembangkan Sistem Monitoring berbasis IoT yang akan bekerja sama dengan salah satu perusahaan swasta yang bergerak di bidang transportasi laut sebagai mitra, yaitu PT Bisma Jaya.

PT Bisma Jaya merupakan perusahaan jasa maritim yang menyediakan berbagai jenis kapal untuk kebutuhan transportasi laut. Berdasarkan informasi yang diperoleh dari Direktur Operasional perusahaan, saat ini digunakan laporan harian dan bulanan sebagai acuan dalam mengestimasi jumlah bahan bakar yang diperlukan di bulan berikutnya. Permasalahan yang umum terjadi adalah waktu sampai yang lebih lama dari estimasi dan sulitnya mengontrol konsumsi bahan. Sistem Monitoring berbasis IoT dinilai cocok untuk menuntaskan permasalahan tersebut, dimana sistem memungkinkan pemantauan data kecepatan mesin dan bahan bakar secara jarak jauh agar sesuai dengan ketentuan yang berlaku, memastikan penggunaan bahan bakar termanfaatkan dengan maksimal.

Untuk merealisasi penelitian ini dibutuhkan lintas disiplin ilmu, yakni elektronika dan komputer. Oleh karena itu, dibutuhkan suatu metode pengembangan sistem yang lebih menekankan kolaborasi serta komunikasi yang baik. Terdapat metode \textit{Waterfall}, yang merupakan proses desain sekuensial yang digunakan dalam proyek pengembangan perangkat lunak. Ini mengikuti perkembangan linier melalui fase yang berbeda, termasuk pengumpulan dan analisis kebutuhan, desain, pengkodean, pengujian, dan pemeliharaan \parencite{article:abbas}. Metode ini, mengasumsikan kebutuhan telah final di awal proyek, serta tiap fase harus diselesaikan terlebih dahlu sebelum lanjut ke fase berikutnya. Oleh karenanya, metode ini tidak cocok untuk diterapkan pada sistem yang memiliki kebutuhan yang dinamis. Sehingga, penggunaan metode \textit{Waterfall} dirasa kurang cocok dalam pengembangan Sistem Monitoring berbasis IoT dikarenakan sulitnya untuk mengatur perubahan dan beradaptasi pada kebutuhan yang berkembang.

\textit{Agile} merupakan pendekatan yang secara efektif dapat beradaptasi dengan kebutuhan yang berubah-ubah, yang mana ini sulit untuk diatur pada model \textit{Waterfall}. Salah satu metode yang populer adalah Scrum. Scrum merupakan metodologi yang berfokus pada pengembangan berulang dan bertahap, fleksibilitas, dan perbaikan terus menerus. Metodologi ini sangat cocok untuk pengembangan proyek skala masif dengan personel pengembang yang banyak.

Lalu terdapat metode \textit{Extreme Programming} (XP), sebuah metode \textit{agile} yang menekankan pada kolaborasi, adaptasi, dan pengembangan iteratif \parencite{article:matharu}. \textit{Extreme Programming} metode yang ideal untuk digunakan tim skala kecil menengah dalam pengembangkan perangkat lunak dengan cepat serta fleksibel dalam menghadapi perubahan. Salah satu prinsip dari metode ini adalah keterlibatan pelanggan \parencite{article:matharu}. Hal ini memastikan perangkat lunak memenuhi kebutuhan pelanggan dan mengurangi risiko pengembangan fitur yang tidak diperlukan.

Berdasarkan pertimbangan pilihan metode yang sudah dilakukan, metode yang paling cocok untuk diterapkan pada studi kasus penelitian ini adalah metode \textit{Extreme Programming} (XP) karena dari perusahaan membutuhkan sistem yang dapat dengan cepat diimplementasikan tanpa harus melalui proses dokumentasi yang banyak. Metode ini cocok untuk pengembangan sistem dengan tim yang sedikit dan dalam kurun waktu yang relatif singkat, serta bersifat fleksibel terhadap perubahan dikarenakan adanya kemungkinan perubahan kebutuhan terkait fitur-fitur yang ada pada sistem.

Diharapkan dengan adanya penelitian ini, Sistem Monitoring pada mesin diesel yang dikembangkan dapat membantu PT Bisma Jaya khususnya Direktur Operasional dalam melakukan pemantauan penggunaan bahan bakar serta menjaga efektivitas armada kapal yang sedang beroperasi yang mulanya melalui laporan harian/bulanan yang dibuat secara manual menjadi sistem yang dapat menyajikan data historis dan dapat diakses kapan saja.



\section{Rumusan Masalah}

Berdasarkan latar belakang dari penelitian, didapatkan rumusan masalah sebagai berikut.

\begin{enumerate}
    \item Bagaimana sistem monitoring dirancang menggunakan metode \textit{Extreme Programming}?
    \item Bagaimana sistem monitoring dikembangkan menggunakan metode \textit{Extreme Programming}?
\end{enumerate}

\section{Tujuan}

Tujuan dari penelitian ini adalah sebagai berikut.

\begin{enumerate}
    \item Untuk merancang sistem monitoring menggunakan metode \textit{Extreme Programming}.
    \item Untuk mengembangkan sistem monitoring menggunakan metode \textit{Extreme Programming}.
\end{enumerate}

\section{Manfaat}

Manfaat yang didapatkan pada penelitian ini adalah sebagai berikut.

\begin{enumerate}
    \item Membantu perusahaan dalam melakukan pengawasan dan kontrol konsumsi bahan bakar armada kapal selama operasi.
    \item Membantu perusahaan dalam memastikan efektivitas operasi
\end{enumerate}


\section{Batasan Penelitian}

Batasan penelitian ini adalah sebagai berikut.

\begin{enumerate}
    \item Fokus utama pada penelitian ini adalah pengembangan Sistem Monitoring berbasis web
    \item Pada penelitian ini diterapkan 3 layer arsitektur IoT teratas: \textit{App Layer}, \textit{Data Processing Layer}, dan \textit{Network Layer}
    \item Sistem Monitoring berbasis IoT dikembangkan menggunakan framework
    NextJS, Django, dan MySQL sebagai \textit{Database Management Systems (DBMS)}
\end{enumerate}
\section{Kerangka Pemikiran Penelitian}

Berikut kerangka pikiran pada penelitian ini.

\begin{figure}[ht]
    \includegraphics[width=1\linewidth, center]{images/pendahuluan/fig-framework-penelitian.jpg}
    \caption{Kerangka Penelitian}
    \label{fig:thinking-framework}
\end{figure}

\newpage

Gambar 1.1  merupakan kerangka pemikiran penelitian yang bertujuan untuk memberikan gambaran mengenai urgensi implementasi Sistem Monitoring berbasis \textit{Internet of Things (IoT)} di PT Bisma Jaya. Perusahaan ini dihadapkan pada masalah utama berupa tingginya pengeluaran untuk bahan bakar selama operasionalnya, yang dipengaruhi oleh sejumlah permasalahan dalam aspek-aspek ekonomi, metode, teknologi, dan manusia.

Kategori ekonomi, laporan harian secara rutin dibuat setiap malamnya oleh tim operasional kapal. Selama operasi, mereka menjaga catatan aktivitas dalam sebuah jurnal yang mencatat waktu perjalanan dan berhenti kapal. Namun, terdapat kekurangan dalam pencatatan yaitu tidak adanya pencatatan jumlah jam operasi pada tiap kategori operasi tertentu. Akibatnya, ketika mereka membuat laporan, nilai \textit{running hour} untuk tiap kategori operasi hanya dapat diestimasikan saja, yang bisa berpotensi mengakibatkan perhitungan konsumsi bahan bakar lebih tinggi dari seharusnya. Informasi lebih lanjut mengenai kategori operasi dapat dilihat pada Bab \ref{ch:2} bagian \ref{sec:fcrv}.

Kategori metode, perusahaan selama ini mengandalkan laporan bulanan yang dibuat tim operasional kapal. Hanya saja, tidak ada data faktual yang dapat dijadikan pembanding terhadap laporan yang dibuat. Selain itu, laporan tersebut baru diterima setiap akhir bulan, sehingga perusahaan tidak memiliki data apapun hingga mendapat temuan dari pihak pengguna.

Kategori teknologi, tidak adanya suatu sistem monitoring yang dipasang pada armada juga menjadi salah satu masalah utama perusahaan. Selama ini, perusahaan hanya mengandalkan data dari AVTS untuk mengetahui data \textit{running hour}, kecepatan (knot), dan posisi kapal. Tetapi, alat ini tidak dapat memberikan informasi detail terkait kecepatan mesin dan konsumsi bahan bakar.

Kategori manusia, pengguna jasa perusahaan telah mengalami situasi di mana mereka melaporkan adanya kejadian kehilangan bahan bakar. Sebelum armada kapal mengisi ulang bahan bakarnya, operator \textit{fuel management} pengguna jasa akan melakukan pengukuran yang dikenal dengan istilah "sounding." Hasil dari pengukuran ini akan dibandingkan dengan laporan harian yang disusun oleh tim operasional kapal. Apabila ditemukan selisih sebesar lebih dari 100 liter, operator \textit{fuel management} akan melaporkan indikasi kehilangan yang dapat mengakibatkan dikenakan denda.

Secara garis besar, didapatkan inti permasalahan yang terjadi di PT Bisma Jaya adalah belum adanya langkah kontrol pada bahan bakar pada kapal yang dimiliki, sehingga pada penelitian ini akan dikembangkan Sistem Monitoring mesin diesel yang akan membantu perusahaan dalam melakukan pemantauan jumlah bahan bakar selama operasi.
    \chapter{TINJAUAN PUSTAKA}

\section{PT Bisma Jaya}

\noindent PT Bisma Jaya merupakan perusahaan yang bergerak di industri transportasi angkutan laut yang berbasis di Balikpapan, Kalimantan Timur. Sejak tahun 2011, perusahaan telah menyediakan berbagai jenis kapal untuk kebutuhan transportasi industri. Dalam menjalankan tugasnya PT Bisma Jaya memiliki struktur organisasi seperti pada Gambar 2.1.

\begin{figure}[!h]
    \includegraphics[width=1\linewidth, center]{images/tinjauan-pustaka/fig-org-structure.jpg}
    \caption{Struktur Organisasi PT Bisma Jaya}
    \label{fig:org-structure}
\end{figure}

Gambar 2.1 memberikan gambaran struktur organisasi dari PT Bisma Jaya yang dipimpin oleh Direktur yang membawahi \textit{Human Resource}, Direktur Operasional, Keuangan dan Akunting, dan Administrasi. Direktur Operasional membawahi beberapa bagian seperti \textit{Field Agent}, \textit{Field Coordinator}, dan \textit{Maintenance}. Dalam penelitian ini, peneliti akan lebih banyak berkomunikasi dengan Direktur Operasional mengenai hal teknis maupun non teknis selama pengembangan sistem monitoring. 

\section{\textit{Internet of Things}}

\noindent \textit{Internet of Things (IoT)} merujuk pada keterhubungan antara obyek, perangkat, mesin satu dengan lainnya dan internet mengizinkan mereka untuk mengumpulkan dan menukar data \parencite{inproc:gazis}. Secara arsitektur IoT dapat dibagi menjadi 4 lapisan utama: \textit{sensing layer}, \textit{network layer}, \textit{data processing layer}, dan \textit{application layer} \parencite{article:sikder}. Detailnya dapat dilihat pada Gambar 2.2

\begin{figure}[ht]
    \includegraphics[width=0.6\linewidth, center]{images/tinjauan-pustaka/fig-iot-architecture.png}
    \caption{Lapisan dan Komponen Arsitektur IoT \parencite{article:sikder}}
    \label{fig:iot-architecture}
\end{figure}

\noindent Berikut gambaran umum pada setiap lapisan:
\begin{enumerate}
    \item \textbf{\textit{Sensing Layer}}
    
    Lapisan ini bertanggung jawab untuk memanfaatkan berbagai sensor dan perangkat untuk mengumpulkan data. Sensor seringkali memberikan data berupa angka mentah seperti tegangan. Oleh karena itu, perangkat IoT dapat memprosesnya terlebih dahulu sebelum dikirim ke server - disebut juga dengan \textit{edge-computing} - atau langsung meneruskan data tersebut ke lapisan jaringan untuk diproses di server.
    

    \item \textbf{\textit{Network Layer}}
        
    Lapisan ini bertugas mengirimkan data yang diperoleh sensor ke lapisan pemrosesan data untuk diolah. Lapisan ini juga bertugas mengawasi bagaimana perangkat jaringan IoT berkomunikasi satu sama lain. Untuk menjaga keamanan komunikasi, digunakan token autentikasi setiap adanya pengiriman data ke \textit{server}.

    \item \textbf{\textit{Data Processing Layer}}
    
    Pemrosesan dan analisis data sensor berada di bawah lingkup lapisan ini. Selain itu, ia bertugas mengelola dan menyimpan data. Lapisan pemrosesan data sangat penting untuk menghasilkan \textit{insight} berharga dan mengambil tindakan yang sesuai berdasarkan data yang dikumpulkan.
    
    \item \textbf{\textit{Application Layer}}
    
    Dengan menggunakan data yang dikumpulkan dan diproses, lapisan ini bertanggung jawab untuk memberikan \textit{actionable insight} kepada pengguna akhir. Ini bertugas memastikan kerahasiaan dan keamanan data yang diproses dan dianalisis dan merupakan lapisan teratas dalam arsitektur IoT. 

\end{enumerate}

\section{Raspberry Pi}

\noindent Raspberry Pi merupakan \textit{single-board computer} (SBC) yang telah mendapatkan perhatian dan popularitas yang signifikan dalam beberapa tahun terakhir. Teknologi inovatif ini memungkinkan berbagai penerapan dan sekarang penting dalam bidang ilmu dan teknik komputer \parencite{article:johnston}. Raspberry Pi adalah pilihan yang bagus untuk aplikasi \textit{Internet of Things} (IoT) karena portabilitasnya, paralelismenya, keterjangkauannya, dan konsumsi dayanya yang rendah \parencite{article:hosny}. Hal yang membuat Raspberry Pi dapat diandalkan sebagai perangkat IoT adalah adanya 40 pin GPIO yang memungkinkan ia dihubungkan ke beragam sensor dengan berbagai \textit{interface}. Raspberry Pi serta informasi GPIO dapat dilihat pada Gambar 2.3.

\begin{figure}[!h]
    \includegraphics[width=0.9\linewidth, center]{images/tinjauan-pustaka/fig-raspy.png}
    \caption{Raspberry Pi dan 40 pin GPIO}
    \label{fig:raspy}
\end{figure}

Setiap pin GPIO dapat digunakan sebagai pin input maupun output, dan dapat digunakan untuk berbagai kebutuhan. Terdapat pin 5v dan 3.3v yang berjumlah masing-masing 2, juga beberapa pin \textit{ground} yang tidak dapat dikonfigurasi. Sisanya merupakan pin \textit{general purpose} 3.3v, yang berarti  output diatur ke 3.3v dan input toleran dengan nilai 3.3v. Pin output dapat diatur ke \textit{high} (3.3v) dan \textit{low} (0v). Begitu juga dengan pin input, dapat membaca \textit{high} (3.3v) dan \textit{low} (0v). Selain itu, pin GPIO juga dapat digunakan untuk kebutuhan yang memerlukan jenis pin yang spesifik seperti \textit{PWM (pulse-width modulation)} untuk membuat sinyal analog; \textit{SPI (serial peripheral interface)} untuk transfer data antar Raspberry Pi dengan perangkat periferal; \textit{I2C (inter-integrated circuit)} untuk komunikasi dengan berbagai jenis sensor; dan Serial untuk pembacaan data serial.

\section{NextJS}

\noindent NextJS merupakan \textit{framework} JavaScript yang menjadi standar pengembangan web modern berbasis JavaScript. JavaScript sendiri merupakan bahasa pemrograman yang sering digunakan khususnya pada pemrograman web. Menurut \textcite{article:tomasdottir}, JavaScript dikenal dengan sifatnya yang dinamis, yang mencakup fitur-fitur seperti \textit{dynamic typing}, \textit{dynamic file loading}, \textit{first-class functions}, dan \textit{property access}. JavaScript juga merupakan bahasa yang \textit{dynamically typed} meskipun sistem \textit{static type} seperti TypeScript dan Flow telah dibuat untuk itu,  \parencite{article:gao}. JavaScript adalah pilihan yang bagus untuk pembuatan aplikasi dengan tampilan data yang dinamis.

\section{Django}

\noindent Django merupakan \textit{framework} Python dalam pengembangan web. Pada penelitian ini Django digunakan sebagai perantara bagi modul IoT dan database berinteraksi melalui API. Python adalah bahasa pemrograman populer yang digunakan secara luas di berbagai bidang. Ia terkenal dengan syntaxnya yang singkat dan sederhana, sehingga cocok untuk otomatisasi proses dan pengintegrasian aplikasi \parencite{article:buhler}. Popularitas Python dapat dikaitkan dengan kemampuan beradaptasi dan ketersediaan berbagai \textit{library} dan \textit{framework} yang mempercepat dan menyederhanakan pengembangan \parencite{article:malloy}.

\section{MySQL}

\noindent MySQL adalah sistem manajemen \textit{database open-source} yang umum digunakan sebagai penghubung perangkat lunak dengan \textit{database} server. MySQL dapat secara efektif mengelola banyak pengguna secara bersamaan dan data dalam jumlah besar \parencite{article:gomez}. Oleh karenanya, database ini cocok digunakan untuk menyimpan log data yang akan diterima dari sensor.

\section{\textit{Entity Relationship Diagram}}

\noindent \textit{Entity relationship diagram} basis data direpresentasikan secara visual dalam diagram hubungan entitas (ERD). Dengan menggunakan metode \textit{top-down}, ini mewakili hubungan antara entitas dan atributnya dan mengatur data berdasarkan informasi semantik \parencite{article:chen}. Untuk memberikan representasi yang jelas dan ringkas tentang struktur dan hubungan dalam database, ERD sering digunakan dalam desain dan pemodelan database \parencite{article:supriyadi}.

\section{Metode Perhitungan Bahan Bakar}

\noindent Dokumen Fuel Consumption Rate Verification (FCRV) merupakan dokumen yang berisi informasi kategori operasi berdasarkan rentang kecepatan tertentu. Ini akan digunakan awak kapal ketika hendak melaporkan jumlah konsumsi bahan bakar berdasarkan jumlah running hour pada rentang angka kecepatan yang telah ditentukan. Berikut contoh isi dari Dokumen FCRV.

\begin{table}[!h]
    \centering
     \begin{tabular}{c c c c} 
        \toprule
        Operation Category & 
        Max Fuel Used (L) & 
        RPM & 
        Average Speed (knot) \\ [0.5ex] 
        \midrule
        Full Speed          & 28    & 1100      & 5 \\  
        Economical Speed    & 18    & 900-1000  & 4 \\  
        Slow Speed/Maneuver & 11    & 700-800   & 3 \\  
        Idle Speed          & 6     & 600       & 0 \\  
        Standby (M/E Off)   & 0     & 0         & 0 \\ [1ex] 
        \bottomrule
     \end{tabular}
     \caption{Fuel Consumption Rate Verification}
     \label{tab:fcrv}
\end{table}

Pada tabel diatas, terdapat 4 kategori operasi yakni Full Speed, Economical Speed, Slow Speed/Manuever, dan Idle Speed. Full Speed adalah kondisi kecepatan mesin tertinggi yang hanya digunakan di laut lepas, nilai maksimum konsumsi bahan bakar (FCR) dalam 1 jam mencapai 28L. Lalu, terdapat Economical Speed. Ini merupakan kategori kecepatan tertinggi kedua dan yang paling sering digunakan ketika menyusuri sungai. Kategori ini memiliki rentang RPM 900-1000 dengan nilai FCR 18L dalam 1 jam. Selanjutnya terdapat Slow Speed, dimana kecepatan ini digunakan untuk mengatur posisi kapal di pelabuhan. Kategori ini memiliki rentang RPM 700-800 dengan nilai FCR 11L dalam 1 jam. Terakhir, Idle Speed dimana kapal dalam kondisi tidak bergera namun mesin menyala. Rentang RPM pada kategori ini adalah 700 kebawah dan hanya memakan bahan bakar 6L dalam 1 jam.

Pada praktiknya, awak kapal hanya mengisi nilai running hour dari masing-masing kategori untuk mendapatan nilai konsumsi bahan bakar. Nilai running hour ini didapatkan berdasarkan estimasi mengikuti jurnal aktivitas/pergerakan kapal. Contoh pengisian tabelnya adalah sebagai berikut.


\begin{table}[!h]
    \centering
     \begin{tabular}{c c c} 
        \toprule
        Running Hour & 
        Operation Category & 
        Fuel Consumption (L) \\ [0.5ex] 
        \midrule
        01:00   & Full Speed            & 28    \\  
        02:30   & Economical Speed      & 45    \\  
        00:30   & Slow Speed/Manuever   & 5.5   \\  
        00:10   & Idle Speed            & 1     \\ [1ex] 
        \bottomrule
     \end{tabular}
     \caption{Contoh Laporan Penggunaan Bahan Bakar}
     \label{tab:fc-report-example}
\end{table}

\section{Perbandingan SDLC}

\noindent Berikut adalah perbandingan dari metodologi \textit{Extreme Programming} dengan salah satu metode sequential, yaitu Waterfall dan metode Agile lainnya, yaitu Scrum menurut \textcite{inproc:fahrurrozi} dan \textcite{article:suryantara}. 

\begin{longtable}[!h]
        {
            p{0.19\textwidth}
            p{0.27\textwidth}
            p{0.27\textwidth}
            p{0.27\textwidth}
        }
        
        \toprule
        Tahapan dalam pengembangan & \textit{Extreme Programming} & Waterfall & Scrum \\ [0.5ex] 
        \midrule
        \textit{Planning} 
        & 
        Pada tahap ini dilakukan pengumpulan kebutuhan sistem dan menjadikannya dalam bentuk user story dan diurutkan berdasarkan tingkat kesulitannya. 
         
        & 
        Tahap ini merupakan langkah awal dimana kebutuhan proyek dikumpulkan dan dianalisis.  
        
        & 
        Tahap ini dibagi menjadi 2 bagian: Sprint \textit{Planning} dan Release \textit{Planning}. Sprint \textit{Planning} dilakukan setiap awal sprint dan Release \textit{Planning} dilakukan setiap awal rilis. 
         
        \\
        \midrule

        \textit{Analysis} 
        & 
        Developer kemudian memutuskan user story apa saja yang akan dikerjakan pada iterasi mendatang 
        & 
        Pada Tahap ini kebutuhan yang dikumpulkan akan dipecahkan menjadi potongan yang dapat dikelola
        
        & 
        Analisis terjadi saat Sprint \textit{Planning}, dimana tim akan memilih perkerjaan yang akan mereka selesaikan di sprint tersebut
        
        \\ 
        \midrule

        \textit{Design} 
        & 
        Tim bekerja dalam iterasi singkat untuk menghasilkan software yang berfungsi, dan desainnya berkembang seiring kemajuan proyek
        & 
        Tahap desain melibatkan pembuatan rencana rinci untuk software berdasarkan persyaratan yang dikumpulkan dalam fase perencanaan dan analisis. 
        & 
        Perancangan terjadi selama Sprint \textit{Planning}, dimana tim memilih pekerjaan yang akan mereka selesaikan selama sprint 
        
        \\ 
        \textit{Implementation} 
        & 
        Developer bekerja dalam waktu singkat untuk menghasilkan software yang berfungsi, dan implementasinya berkembang seiring kemajuan proyek
        & 
        Tahap implementasi melibatkan koding software berdasarkan rencana rinci yang dibuat pada fase desain 
        & 
        Implementasi terjadi selama sprint, dimana tim menyelesaikan pekerjaan yang mereka pilih selama Sprint \textit{Planning} 
        
        \\ 
        \midrule

        \textit{Support \& Security} 
        & 
        Support dan security adalah proses yang berlangsung sepanjang proyek. Tim bekerja dalam waktu singkat untuk menghasilkan software yang berfungsi, dan perangkat lunak tersebut terus diperbarui dan dipelihara
        & 
        Tahap support dan security terjadi setelah software dikirimkan. Tahap ini melibatkan pemeliharaan dan pembaruan perangkat lunak untuk memastikannya terus memenuhi kebutuhan pengguna 
        & 
        Support dan security terjadi setelah perangkat lunak dikirimkan. Fase ini melibatkan pemeliharaan dan pembaruan perangkat lunak untuk memastikannya terus memenuhi kebutuhan pengguna 
        \\ [1ex] 

        \bottomrule
    \caption{Perbandingan metodologi SDLC}
    \label{tab:sdlc-comparison}
\end{longtable}

\section{\textit{Extreme Programming} (XP)}

\noindent \textit{Extreme Programming} (XP) adalah pendekatan \textit{agile software development} yang memberikan penekanan pada kerja sama, pengembangan iteratif dan berulang, serta kemampuan beradaptasi terhadap perubahan kebutuhan. XP merupakan metodologi sederhana yang dibuat untuk tim pengembang kecil yang bertujuan untuk meningkatkan kualitas dan produktivitas perangkat lunak \parencite{article:matharu}. Kesulitan yang ditimbulkan oleh siklus pengembangan yang panjang dalam praktik pengembangan perangkat lunak konvensional menyebabkan terciptanya XP \parencite{article:rao}.

Ada berbagai prinsip dasar yang mendefinisikan XP. \textit{Continuous planning}, yang memerlukan komunikasi dan kolaborasi rutin antara pengembang dan pemangku kepentingan untuk memastikan bahwa tujuan dan persyaratan proyek dipahami dan dipenuhi \parencite{article:matharu}. Siklus hidup pada metode \textit{Extreme Programming} meliputi \textit{Exploration Phase}, \textit{Planning Phase}, \textit{Iteration to Release Phase}, \textit{Productionizing Phase}, \textit{Maintenance Phase}, dan \textit{Death Phase}. Lebih lengkapnya dapat dilihat pada gambar berikut.

\begin{figure}[ht]
    \includegraphics[width=1\linewidth, center]{images/tinjauan-pustaka/fig-xp-lifecycle.png}
    \caption{Siklus Hidup Metode \textit{Extreme Programming} \parencite{article:anwer}}
    \label{fig:xp-lifecycle}
\end{figure}

Berikut penjalasan detail untuk setiap fase:
\begin{enumerate}
    \item \textbf{Exploration Phase:}
    Pada fase ini, dihasilkan user story yang dibuat berdasarkan hasil pengambilan data baik dari observasi, interview, dan dialog dengan mitra. User story ini dapat bertambah seiring waktu mengikuti kebutuhan mitra.

    \item \textbf{Planning Phase:}
    Selanjutnya, user story yang sebelumnya dibuat akan dikumpulkan dan diprioritaskan berdasarkan perhitungan poin story. Ini akan membantu kita dalam menentukan user story mana yang akan dikerjakan pada iterasi berikutnya.
    
    \item \textbf{Iteration to Release Phase:}
    Tahap iterasi merupakan tahap dimana pengembang akan mengimplementasi sistem berdasarkan user story yang ditentukan. Pertama, dilakukan tahap analisis untuk mengonversi kebutuhan mitra menjadi user flow untuk desain tampilan dan algoritma untuk logika sistem. Lalu, dilakukan desain tampilan sesuai dengan user flow yang dihasilkan dan dilakukan perencanaan untuk pengujian. Terakhir, dilakukan pengujian oleh pengembang sebelum kode diunggah ke repositori. Proses programming dilakukan secara parallel dari tahap analisis hingga pengujian. Setelah sistem berhasil melewati unit test dan integration test, sistem akan diuji oleh mitra dan hanya dapat lanjut ke tahap berikutnya setelah mendapatkan persetujuan.
    
    \item \textbf{Productionizing Phase:}
    Sistem yang telah diunggah di repositori akan diluncurkan di server dengan mode development. Ini memungkinkan mitra untuk melakukan pengujian fitur yang masih dalam proses persetujuan serta memberikan umpan balik secara berkala. 
    
    \item \textbf{Maintenance Phase:}
    Iterasi yang mendapatkan persetujuan selanjutnya akan diluncurkan pada tahap ini. Dapat dikatakan sistem yang terdapat pada tahap ini merupakan gambaran terakhir dari sistem secara keseluruhan.
    
    \item \textbf{Death Phase:}
    Ini merupakan tahap terakhir dimana sistem akan diluncurkan secara penuh di server dengan mode production.
\end{enumerate}

Melihat dari tantangan industri mitra yang dinamis serta perlunya kolaborasi yang kuat dari berbagai lintas disiplin ilmu untuk mewujudkan Sistem Monitoring berbasis IoT ini, diputuskanlah metode Extreme Programming sebagai metodologi pengembangan perangkat lunak yang menekankan pada komunikasi dan kolaborasi serta sifatnya yang agile memungkinkan pengembang untuk menjawab berbagai tantangan industri tanpa interupsi selama proses pengembangan sistem.

\section{Penelitian Terdahulu}

\noindent Berikut rangkuman hasil penelitian terdahulu yang memiliki keterkaitan dengan penelitian yang dilakukan.

\begin{longtable}[!h]
        {
            p{0.05\textwidth}
            p{0.3\textwidth}
            p{0.65\textwidth}
        } 
        \toprule
        No & 
        Nama Peneliti dan Tahun & 
        Penelitian yang dilakukan \\ [0.5ex] 
        \midrule
        
        1
        & \textcite{inproc:abdulmalek}
        &
        \textbf{Judul:}
        \textit{IoT-Based Healthcare-Monitoring System towards Improving Quality of Life: A Review}

        \textbf{Permasalahan:}
        Kelemahan utama dari layanan kesehatan adalah hanya tersedia di rumah sakit, sehingga tidak memadai dan terkadang tidak mampu memenuhi kebutuhan lansia dan penyandang disabilitas. Pemantauan status kesehatan lansia secara real-time adalah masalah yang diselesaikan secara efektif dan praktis oleh \textit{Internet of Things} (IoT) dengan penggunaan data sensor dan telekomunikasi.

        \textbf{Hasil:}
        Sistem kesehatan berbasis IoT memfasilitasi hidup orang dalam banyak cara.
        
        \begin{enumerate}
            \item \textbf{Remote healthcare:}
            Daripada pasien mendatangi layanan kesehatan, solusi nirkabel berbasis IoT menghadirkan layanan kesehatan kepada pasien. Sensor berbasis IoT digunakan untuk mengumpulkan data dengan aman, yang kemudian diproses oleh algoritma kecil dan dibagikan kepada penyedia layanan kesehatan untuk mendapatkan rekomendasi yang tepat.

            \item \textbf{Realtime monitoring:}
            Sensor pemantauan berbasis IoT mengumpulkan serangkaian data psikologis. Penyimpanan data dikelola melalui analisis dan \textit{gateway} berbasis \textit{cloud}.
    
    
            \item \textbf{Preventive care:}
            Data sensor digunakan oleh sistem layanan kesehatan IoT untuk memberi tahu anggota keluarga dan membantu deteksi dini keadaan darurat. \textit{Internet of Things} memungkinkan machine learning untuk deteksi anomali dini dan pelacakan tren kesehatan.
        \end{enumerate}
        
        \\

        2
        & \textcite{article:anh}
        & 
        \textbf{Judul:} 
        \textit{Development and Implementation of a low-cost IoT System for Small Farm Households}

        \textbf{Permasalahan:}
        Pertanian kecil memiliki peran yang yang penting untuk produksi agrikultural terutama pada negara kurang berkembang maupun berkembang. Berbeda dengan pertanian skala besar yang berinvestasi pada teknologi mutakhir untuk memastikan kualitas hasil panen yang maksimal, teknologi pada pertanian kecil masih sangat terbatas. 

        \textbf{Hasil:}
        Sistem IoT diusulkan untuk dapat membantu petani kecil meningkatkan kualitas produk pertanian sekaligus mengurangi biaya produksi dan mencegah pemborosan air irigasi dan pupuk. Agrikultur sangat bergantung pada cuaca dan iklim, seperti temperatur dan kadar air tanah. Dalam penelitian, sistem melakukan monitoring pada temperatur, kelembapan, intensitas cahaya, dan kadar air tanah. Parameter tersebut digunakan sebagai acuan dalam mengatur pompa embun, pompa irigasi, jendela ventilasi, kipas ventilasi, dan grow light. 

        \\ 
        
        3
        & \textcite{inproc:hizbullah} 
        &   
        \textbf{Judul:} \textit{Internet of Things} for Smart Transportation in North Moluccas Province

        \textbf{Permasalahan:} Perlunya transportasi yang lebih aman dan penyediaan layanan keselamatan selama keadaan darurat di wilayah Provinsi Maluku Utara.

        \textbf{Hasil:} Diterapkan otomasi pada sistem navigasi yang dapat membantu meningkatkan akurasi dan keandalan navigasi perahu agar mengurangi risiko kecelakaan. Peneliti juga menerapkan sistem monitoring yang dapat menyediakan data secara real-time terhadap kondisi perahu untuk kebutuhan maintenance dan deteksi lebih awal isu yang mungkin akan terjadi.
        
        \\ 
        \midrule

        4
        & \textcite{article:maswadi} 
        &   
        \textbf{Judul:} \textit{Systematic Literature Review of Smart Home Monitoring Technologies Based on IoT for the Elderly}

        \textbf{Permasalahan:} Dengan seiring bertambahnya populasi lansia berumur 65 keatas di negara-negara seperti Amerika, Jerman, Perancis, Itali, dan Jepang terdapat kemungkinan mereka akan beban yang bertambah pada kesehatan dan layanan sosial. Diperlukan teknologi yang dapat memberikan lingkungan hidup yang kondusif bagi para lansia.

        \textbf{Hasil:} Penerapan teknologi sistem smart home pada lansia telah secara signifikan meningkatkan kualitas hidup diantara para lansia. Beberapa teknologi yang dilaporkan telah menyelamatkan hidup para lansia di situasi darurat. 
        
        \\ 

        5
        & \textcite{article:song} 
        &   
        \textbf{Judul:} \textit{Internet of Maritime Things Platform for Remote Marine Water Quality Monitoring}

        \textbf{Permasalahan:} Penerapan sistem monitoring kualitas air di laut memerlukan dukungan komunikasi jarak jauh dan berkecepatan tinggi yang stabil.

        \textbf{Hasil:} Dalam penelitian ini, dikembangkan sebuah platform IoT Maritim yang mendukung komunikasi jarak jauh dan berkecepatan tinggi untuk pemantauan kualitas air laut jarak jauh dan online. Perangkat ditempatkan di atas permukaan air laut dan gerbang untuk pengiriman data ditempatkan darat. Untuk merealisasi komunikasi jarak jauh dan berkecapatan tinggi antara perangkat dengan control center di darat, dikembangkan sistem penyesuaian sinar otomatis (automatic beam adjustment system) untuk antena pengarah sehingga dapat mendukung komunikasi jarak jauh dan berkecepatan tinggi dengan secara otomatis mengatur derajat sinar agar selalu mengarah ke gateway di darat. Metode ini terbukti memberikan performa komunikasi dua kali lipat dibandingkan koneksi nirkabel (LTE di laut) yang ada.
        
        \\
        \bottomrule
     \caption{Penelitian terdahulu mengenai \textit{Internet of Things} (IoT)}
     \label{tab:prev-research}
\end{longtable}
    \chapter{METODE PENENILITIAN}

\section{Garis Besar Penelitian}

\noindent Secara garis besar, penelitian ini akan melaksanakan pengembangan sistem monitoring mesin diesel berbasis \textit{IoT} yang akan diuji pada PT Bisma Jaya untuk meningkatkan efektivitas operasi dengan menekankan kecepatan minimum pada armada dan efisiensi biaya bahan bakar dengan memantau jumlah bahan bakar yang digunakan setiap harinya. Tampilan web akan dikembangkan menggunakan framework NextJS berbasis JavaScript dan sistem \textit{backend} menggunakan framework Django berbasis Python. Pengembangan sistem ini akan menggunakan metode Extreme Programming yang memiliki tahapan sebagai berikut: Exploration Phase, Planning Phase, Iteration to Release Phase, Productionizing Phase, Maintenance Phase, dan Death Phase. 

\section{Diagram Alir Penelitian}

\noindent Metodologi pelaksanaan penelitian ini dapat dimodelkan menggunakan diagram alir sebagai berikut.

\begin{figure}[!h]
    \includegraphics[width=1.1\linewidth, center]{images/metode/flowchart-penelitian.jpg}
    \caption{Diagram Alir Penelitian}
    \label{fig:flow-research}
\end{figure}


Pada Gambar 3.1 dapat dilihat bahwa penelitian ini diawali dengan melakukan identifikasi masalah dan studi literatur mengenai jurnal-jurnal dari penelitian relevan yang telah dilakukan sebelumnya. Kemudian, dilanjutkan dengan pengembangan sistem menggunakan metode \textit{Exteme Programming (XP)} yang terdiri dari Exploration Phase, Planning Phase, Iteration to Release Phase, Productionizing Phase, Maintenance Phase, dan Death Phase. Terakhir dilakukan pembuatan kesimpulan dan saran dari penelitian.

\section{Prosedur Penelitian}

\noindent Berikut penjelasan prosedur penelitian secara rinci berdasarkan tahapan-tahapan yang ditetapkan pada Gambar 3.1 sebagai pedoman penelitian ini.

\begin{enumerate}
    \item \textbf{Identifikasi Masalah}
    
    Penelitian ini dimulai dengan mengidentifikasi permasalahan yang ada di mitra. Ini dilakukan melalui dua cara, diskusi dengan seluruh \textit{stakeholder} yang terlibat dan observasi ke lapangan. Dari tahap ini akan dihasilkan rumusan masalah, tujuan, serta batasan-batasan penelitian.

    \item \textbf{Studi Literatur}
    
    Pada tahap ini dilakukan studi literatur dengan mengumpulkan berbagai referensi yang didapat dari jurnal, artikel ilmiah, dan sumber lainnya mengenai pengembangan sistem berbasis \textit{IoT}, teknologi-teknologi yang digunakan selama penelitian, dan perbandingan metodologi dalam pengembangan perangkat lunak.

    \item \textbf{Pengembangan Sistem}
    
    \begin{enumerate}[label*=\arabic*.]
        \item \textbf{Perancangan Arsitetur}
        \item \textbf{Extreme Programming}
        
        \begin{enumerate}[label*=\arabic*.]
            \item \textbf{\textit{Exploration Phase}}
            
            Kebutuhan yang dikumpulkan pada tahap identifikasi masalah kemudian dibuat dalam bentuk \textit{user story} untuk mendeskripsikan hasil yang diinginkan. Daftar \textit{user story} sementara dapat dilihat pada tabel berikut.

            \begin{longtable}[!h]
                {
                    p{0.15\textwidth}
                    p{0.15\textwidth}
                    p{0.35\textwidth}
                    p{0.35\textwidth}
                }
                    \toprule
                    \textit{Code} & 
                    \textit{Persona} & 
                    \textit{I want to} &
                    \textit{So that can} \\ [0.5ex] 
                    \midrule
                    
                    US-01 & 
                    \textit{User} & 
                    \textit{Login} & 
                    Mengakses sistem sesuai dengan username dan password 
                    \\  
                    
                    US-02 & 
                    \textit{User} & 
                    \textit{Logout} & 
                    Keluar dari sistem melalui tombol logout \\  
                    
                    US-03 & 
                    \textit{User} & 
                    Melihat data historis kecepatan mesin & 
                    Memastikan armada bergerak dengan kecepatan mesin yang sesuai
                    \\  
                    
                    US-04 & 
                    \textit{User} & 
                    Melihat data historis konsumsi bahan bakar &   
                    Melakukan kontrol bahan bakar
                    \\  
                    
                    US-05 & 
                    \textit{User} &     
                    Melihat data historis \textit{running hour} & 
                    Memastikan operasi berjalan dengan optimal
                    \\  
                    
                    US-06 & 
                    \textit{User} &         
                    Melihat data log kecepatan per menit &   
                    Melihat detail kecepatan pada waktu spesifik
                    \\  
                    
                    US-07 & 
                    \textit{User} &       
                    Mencetak laporan kecepatan mesin harian &     
                    Mendapatkan laporan harian kecepatan mesin dengan format PDF
                    \\  
                    
                    US-08 & 
                    \textit{User} &     
                    Mencetak laporan harian konsumsi bahan bakar & 
                    Mendapatkan laporan harian konsumsi bahan baar dengan format PDF 
                    \\  
                    
                    US-09 & 
                    \textit{User} &         
                    Mengunduh data log kecepatan mesin &     
                    Mendapatkan log kecepatan mesin dengan format CSV
                    \\ [1ex] 
                    \bottomrule
                \caption{User Story Sementara}
                \label{tab:user-story}
            \end{longtable}

            \item \textbf{\textit{Planning Phase}}
            
            User story yang dibuat pada tahap sebelumnya akan dikumpulkan dan disimpan berdasarkan prioritas di release plan. Tahap ini akan diulang kembali setelah melewati tahap testing sesuai dengan jumlah iterasi pengembangan.
            
            

            \item \textbf{\textit{Iteration to Release Phase}}
            
            Berdasarkan \textit{user story} yang dibuat di tahap sebelumnya, akan dilakukan perancangan \textit{user interface} sistem dan skema basis data dalam bentuk \textit{entity relationship diagram (ERD)} yang akan membantu dalam menggambarkan hubungan antar tabel. ERD sistem sementara dapat dilihat pada gambar berikut.

            \begin{figure}[!h]
                \includegraphics[width=1\linewidth, center]{images/metode/erd.png}
                \caption{Entity Relationship Diagram (ERD) Sementara}
                \label{fig:erd}
            \end{figure}

            Secara parallel, dilakukan tahap implementasi dari perancangan sistem. Dalam pengembangan dashboard digunakan framework NextJS untuk menghasilkan tampilan yang dinamis, kemudian digunakan framework Django yang berbasis bahasa Python untuk menghubungkan perangkat IoT dengan server. Setelah dilakukannya pengembangan pada satu iterasi, sistem akan diunggah ke repositori dan akan diuji setelahnya oleh mitra melalui metode black box testing. Jika terdapat umpan balik, maka tahapan akan diulang ke tahap desain untuk menyesuaikan permintaan mitra. Sebaliknya, jika mendapatkan persetujuan maka sistem akan masuk ke tahap berikutnya.

            \item \textbf{\textit{Productionizing Phase}}
            
            Setelah melewati pengujian dan mendapatan persetujuan mitra, sistem akan diluncurkan secara perlahan pada server dengan mode development. Fitur yang dirasa kurang lengkap atau ingin ditambahkan oleh mitra akan balik di tahap planning untuk dilakukan prioritas. Jika mendapatkan persetujuan, akan dilanjutkan ke tahap berikutnya.

            \item \textbf{\textit{Maintenance Phase}}
            
            Pada tahap ini, sistem sudah hampir matang dan menunggu persetujuan terakhir dari mitra. Jika ditemukan bug atau ketidak sesuaian pada sistem maka akan kembali ke tahap Iteration to Release.

            \item \textbf{\textit{Death Phase}}
            
            Ini merupakan tahap terakhir, apabila seluruh fitur pada sistem telah selesai dikembangkan dan sesuai dengan kebutuhan awal mitra maka dilakukan tahap peluncuran dimana akan digunakan VPS cloud hosting.
        
        \end{enumerate}
    \end{enumerate}
\end{enumerate}

\section{Rencana Jadwal Penelitian}

\noindent Berikut jadwal penelitian yang disusun berdasarkan metodologi yang telah dijelaskan.

\begin{figure}[!h]
    \includegraphics[width=1.2\linewidth, center]{images/metode/schedule.png}
    \caption{Rencana Jadwal Penelitian}
    \label{fig:flow-schedule}
\end{figure}
    \chapter{HASIL DAN PEMBAHASAN}

\section{Perancangan (Exploration Phase)}

\subsection{Requirements}

Pada fase ini, seluruh kebutuhan pengguna akan dikumpulkan dalam bentuk user story. Seluruh user story akan diubah menjadi task yang dapat dikerjakan pada jangka waktu tertentu yang disebut dengan iterasi. Pada metode Extreme Programming, iterasi dapat berlangsung selama 1 hingga 2 pekan. Dalam penelitian ini, 1 iterasi akan ditetapkan berlangsung selama 1 pekan. User story pada penelitian ini dapat dilihat pada tabel dibawah.

\begin{longtable}[!h]
    {
            p{0.1\textwidth}
            p{0.1\textwidth}
            p{0.35\textwidth}
            p{0.35\textwidth}
    }
    \caption{Daftar \textit{user story} Sistem Monitoring}
    \label{tab:user-story-fix} \\

    \hline
        \bfseries \textit{Code} &
        \bfseries \textit{Persona} &
        \bfseries \textit{I want to} &
        \bfseries \textit{So that can} \\ [0.5ex]
    \hline

    \endfirsthead

    \hline
        \bfseries \textit{Code} &
        \bfseries \textit{Persona} &
        \bfseries \textit{I want to} &
        \bfseries \textit{So that can} \\ [0.5ex]
    \hline
    \endhead % all the lines above this will be repeated on every page
    \hline

    \csvreader[
        late after line=\\,
        before reading={\catcode`\#=12},after reading={\catcode`\#=6}
    ]{tables/hasil/user-story.csv}{1=\K, 2=\P, 3=\I, 4=\S}{\K & \P & \I & \S} \\

    \bottomrule
\end{longtable}

\subsection{Architecture Modelling}

Dalam buku Fundamental of Software Architecture oleh (Mark Richards dan Neal Ford, 2020), arsitektur didefinisikan sebagai sebuah rangkaian dari keputusan yang akan menentukan struktur dan perilaku sistem. Pada metode XP, model arsitektur dibuat untuk mempertimbangkan berbagai solusi alternatif. Model arsitektur pada Sistem Monitoring dapat dilihat pada Gambar \ref{fig:archi-model-sm} .

\begin{figure}[!h]
    \includegraphics[width=.4\linewidth, center]{images/hasil/archi-model.png}
    \caption{Metafor Model Arsitektur Sistem Monitoring}
    \label{fig:archi-model-sm}
\end{figure}

Gambar di atas merupakan model arsitektur dengan style Service-based Architecture dengan pertimbangan bahwa sistem tidak hanya menyediakan gerbang api (API Gateway) untuk web dasbor saja, melainkan juga untuk perangkat IoT yang akan dipasang di kapal, disebut juga dengan 'IoT Nodes'. IoT Nodes dapat mengirimkan data dengan mengirimkan HTTP Request ke server melalui API Gateway yang telah diatur pada Module Service untuk meneruskan data ke database. Data yang telah tersimpan dapat diakses melalui Web UI yang terhubung dengan System Service melalui API Gateway. Selain mengakses data, sistem juga memungkinkan pengguna untuk dapat melakukan filter data dengan memberi input tanggal yang ditunjukkan oleh dua garis penghubung pada setiap obyek yang menghubungkan Web UI dengan Database.

\subsection{Tools dan Teknologi}

Selama pengembangan sistem, digunakan beberapa tools yang akan membantu proses tersebut serta pemilihan teknologi (techstack) untuk Sistem Monitoring yang sebelumnya telah dibahas pada Bab 2 dan Bab 3. Untuk tools yang digunakan dalam penelitian ini meliputi VSCode sebagai IDE, phpmyadmin untuk melakukan manajemen basis data, Figma untuk mendesain arsitektur, skema basis data, dan prototype, serta Apidog untuk pengujian API.

\section{Perencanaan \textit{(Planning)}}

\subsection{Rencana Awal \textit{(Initial Planning)}}

Pada tahap ini, seluruh User Story akan menjadi kumpulan task yang akan disortir berdasarkan tingkat prioritas yang dinilai dari 1 hingga 5 dan estimasi waktu pengerjaannya dalam satuan hari. Seluruh rencana iterasi dapat dilihat pada Tabel \ref{tab:iteration-1} hingga Tabel \ref{tab:iteration-5} dibawah.

Iterasi pertama difokuskan untuk mengelola data kecepatan mesin dan menampilkannya dalam bentuk grafik.

\begin{longtable}[!h]
    {
            p{0.15\textwidth}
            p{0.4\textwidth}
            >{\centering\arraybackslash}p{0.15\textwidth}
            >{\centering\arraybackslash}p{0.15\textwidth}
    }
    \caption{Rencana Iterasi 1}
    \label{tab:iteration-1} \\

    \hline
        \bfseries Kode User Story &
        \bfseries Deskripsi Task &
        \bfseries Prioritas &
        \bfseries Estimasi Waktu \\ [0.5ex]
    \hline

    \endfirsthead

    \hline
        \bfseries Kode User Story &
        \bfseries Deskripsi &
        \bfseries Prioritas &
        \bfseries Estimasi Waktu \\ [0.5ex]
    \hline
    \endhead % all the lines above this will be repeated on every page
    \hline

    \csvreader[
        late after line=\\,
        before reading={\catcode`\#=12},after reading={\catcode`\#=6}
    ]{tables/hasil/iterations/1/task.csv}
    {1=\c, 2=\d, 3=\p, 4=\t}{\c & \d & \p & \t} \\

    \bottomrule
\end{longtable}

Pada iterasi kedua, dilanjutkan pengembangan halaman Fuel Consumption, Running Hour, dan Data Log.

\input{tables/hasil/initial-planning/iteration2.tex}

Selanjutnya, pada iterasi ketiga dilakukan pengembangan sistem admin yang memungkinkan pengguna melakukan manajemen data seperti data pengguna, data kapal, dan data batas kecepatan pada setiap kategori operasional FCRV.

\input{tables/hasil/initial-planning/iteration3.tex}

Iterasi keempat berfokus pada pembuatan laporan harian kecepatan mesin dan konsumsi bahan bakar dalam format PDF.

\begin{longtable}[!h]
    {
            p{0.2\textwidth}
            p{0.4\textwidth}
            >{\centering\arraybackslash}p{0.15\textwidth}
            >{\centering\arraybackslash}p{0.15\textwidth}
    }
    \caption{Rencana Iterasi 4}
    \label{tab:iteration-4} \\

    \hline
        \bfseries \textit{Kode User Story} &
        \bfseries \textit{Deskripsi Task} &
        \bfseries \textit{Prioritas} &
        \bfseries \textit{Estimasi Waktu} \\ [0.5ex]
    \hline

    \endfirsthead

    \hline
        \bfseries \textit{Kode User Story} &
        \bfseries \textit{Deskripsi} &
        \bfseries \textit{Prioritas} &
        \bfseries \textit{Estimasi Waktu} \\ [0.5ex]
    \hline
    \endhead % all the lines above this will be repeated on every page
    \hline

    \csvreader[
        late after line=\\,
        before reading={\catcode`\#=12},after reading={\catcode`\#=6}
    ]{tables/hasil/iterations/4/task.csv}
    {1=\K, 2=\D, 3=\P, 4=\T}{\K & \D & \P & \T} \\

    \bottomrule
\end{longtable}

Terakhir, pada iterasi kelima dilakukan pengembangan sistem autentikasi dan juga ekspor data mentah kecepatan mesin dalam format CSV.

\input{tables/hasil/initial-planning/iteration5.tex}

\subsection{Perubahan \textit{(Changes)}}

Selama pengerjaan berlangsung, terdapat beberapa umpan balik dari mitra yang mengakibatkan penambahan pada task. Sehingga, ini juga berdampak pada rencana iterasi yang sebelumnya telah dibuat. Secara garis besar, task yang bertambah adalah sistem admin untuk melakukan manajemen data kapal, manajemen data pengguna, manajemen data FCRV. Hasil akhir, terdapat task dengan total sebanyak 13 item yang dikerjakan selama 5 iterasi.

\begin{longtable}[!h]
    {
            p{0.05\textwidth}
            p{0.6\textwidth}
            p{0.1\textwidth}
            p{0.1\textwidth}
    }
    \caption{Umpan Balik Selama Pengembangan}
    \label{tab:feedback} \\

    \hline
        \bfseries No &
        \bfseries Umpan Balik &
        \bfseries Iterasi &
        \bfseries Status \\ [0.5ex]
    \hline

    \endfirsthead

    \hline
        \bfseries No &
        \bfseries Umpan Balik &
        \bfseries Iterasi &
        \bfseries Status \\ [0.5ex]
    \hline
    \endhead % all the lines above this will be repeated on every page
    \hline

    \csvreader[
        late after line=\\,
        before reading={\catcode`\#=12},after reading={\catcode`\#=6}
    ]{tables/hasil/changes.csv}{1=\no, 2=\feedback, 3=\i, 4=\status}{\no & \feedback & \i & \status} \\

    \bottomrule
\end{longtable}

Setelah mendapat kebutuhan awal, terdapat permintaan tambahan untuk membangun sistem admin agar mitra dapat lebih leluasa untuk melakukan konfigurasi data. Sistem admin meliputi manajemen data pengguna, data kapal, dan data FCRV. Setelah direkap di user story dan dilakukan skala prioritas, pengembagan Sistem Admin akan dilakukan pada iterasi 3.
Pada iterasi kedua, terdapat permintaan untuk mencantumkan angka running hour pada halaman engine speed yang diposisikan dibawah angka kecepatan mesin seperti terlihat pada Gambar … . Sedangkan pada iterasi 5, terdapat permintaan untuk menambah 2 halaman, yakni Home dan OP41 Report. Halaman Home berisi seluruh kapal yang terdaftar dan Halaman OP41 Report berisi laporan konsumsi bahan bakar berdasarkan perhitungan dari kategori FCRV yang dapat dilihat pada Gambar ... dan Gambar ... secara berturut-turut.


\section{Implementasi \textit{(Iteration to Release)}}

Pada bagian ini, akan dijelaskan secara detail dari pengembangan sistem mulai dari Iterasi 1 hingga Iterasi 5 dan dilanjut dengan fase validasi melalui User Acceptance Test.

\subsection{Iterasi 1}

\subsubsection{Analisis}

Pada tahap ini akan menghasilkan kebutuhan sistem. Kebutuhan sistem merupakan analisis yang dilakukan untuk mengetahui kebutuhan fungsional sistem. Kebutuhan fungsional sistem sendiri merupakan fungsionalitas yang harus tersedia di sistem sesuai kebutuhan stakeholder yang tertuang di user story. Aturan penomoran kebutuhan sistem dapat dilihat pada Tabel \ref{tab:aturan-penomoran} dan kebutuhan fugsional sistem pada Gambar \ref{fig:fr-es}.

\begin{table}[!h]
    \caption{Aturan Penomoran Kebutuhan Sistem}
    \centering
    \begin{tabular}
        {
            >{\centering\arraybackslash}p{0.2\textwidth}
            >{\centering\arraybackslash}p{0.4\textwidth}
        }
        \toprule

        Kode &
        Keterangan \\ [1ex]

        \midrule

        SM- & Sistem Monitoring \\
        -F- & Kebutuhan Fungsional \\
        -\{x\} & Nomor Urutan Kode Kebutuhan \\

        \bottomrule
    \end{tabular}
    \label{tab:aturan-penomoran}
\end{table}

\begin{figure}[!h]
    \includegraphics[width=.8\linewidth, center]{images/hasil/iterations/1/fr-es.png}
    \caption{Kebutuhan Fungsional Engine Speed}
    \label{fig:fr-es}
\end{figure}

\newpage

\subsubsection{Desain}
Berikut merupakan desain \textit{wireframe low fidelity} pada halaman Engine Speed. Pada halaman ini, pengguna dapat melihat angka kecepatan mesin terakhir dan nilai rata-rata dengan interval 1 jam dalam jangka waktu satu hari yang disajikan dalam bentuk grafik. Jangka waktu dapat dipilih melalui filter yang akan disediakan pada area komponen grafik. Seluruh navigasi pada sistem dapat dilakukan dengan memilih komponen Navbar atau Sidebar.

\begin{figure}[!h]
    \includegraphics[width=1.05\linewidth, center]{images/hasil/iterations/1/lofi-es.png}
    \caption{Wireframe Halaman Engine Speed}
    \label{fig:lofi-es}
\end{figure}

\subsubsection{Coding}

Berdasarkan desain low fidelity pada tahap desain maka dilakukan implementasi pengkodean tampilan halaman Engine Speed yang dapat dilihat pada Gambar \ref{fig:fe-es}. Halaman Engine Speed merupakan halaman untuk melihat kecepatan mesin terakhir dan histori tren dari kecepatan mesin yang disajikan dalam bentuk grafik garis. Pengguna dapat melakukan filter tanggal untuk mendapatkan data sesuai dengan batas waktu yang ditentukan.

\begin{figure}[!h]
    \includegraphics[width=1.05\linewidth, center]{images/hasil/iterations/1/fe-es.png}
    \caption{Frontend Halaman Engine Speed}
    \label{fig:fe-es}
\end{figure}

Halaman ini sepenuhnya menggunakan Client Side Rendering (CSR). Hal ini memungkinkan pengguna untuk melakukan filter data tanpa perlu memuat ulang halaman Engine Speed sehingga memaksimalkan pengalaman penggunaan sistem.

\newpage

\subsubsection{Whitebox Testing}
Pengujian dilakukan dengan membuat Unit Test yang akan dijalankan selama pengembangan sebelum repositori akhirnya diunggah ke codebase atau GitHub. Unit test dilakukan pada pengembagan API yang merupakan bagian dari Data Processing Layer dan Network Layer. Hal ini agar data yang akan ditampilkan pada antarmuka sistem merupakan data yang valid sehingga dapat memberikan wawasan yang tepat. Unit test yang dibuat dapat dilihat pada Gambar ... .

\begin{landscape}

    \subsubsection{Blackbox Testing}

    Setelah repositori telah terupdate, mitra dapat langsung mengakses sistem untuk menguji fungsionalitas fitur yang telah dikerjakan pada iterasi tersebut.

    \begin{table}[!h]
    \caption{Aturan Penomoran Blackbox Testing}
    \centering
    \begin{tabular}
        {
            >{\centering\arraybackslash}p{0.2\textwidth}
            >{\centering\arraybackslash}p{0.4\textwidth}
        }
        \toprule

        Kode &
        Keterangan \\ [1ex]

        \midrule

        SM- & Sistem Monitoring \\
        -T- & Testing \\
        -\{x\}- & Singkatan Nama Fitur \\
        -\{n\} & Nomor Urutan Test Case \\

        \bottomrule
    \end{tabular}
    \label{tab:nr-blackbox}
\end{table}

    Test case yang dibuat untuk halaman Engine Speed memastikan grafik telah ditampilkan dengan sesuai dan ketika pengguna melakukan filter tanggal, data grafik akan terubah sesuai dengan data tanggal yang diinput. Berikut hasil test case untuk iterasi 1

    \begin{longtable}[!h]
    {
            p{0.2\textwidth}
            p{0.3\textwidth}
            p{0.3\textwidth}
            p{0.3\textwidth}
            p{0.3\textwidth}
            p{0.15\textwidth}
    }
    \caption{Blackbox Testing Halaman Engine Speed}
    \label{tab:it1-blackbox-es} \\

    \hline
        \bfseries \textit{Test Code} &
        \bfseries \textit{Test Case} &
        \bfseries \textit{Test Steps} &
        \bfseries \textit{Expected Result} &
        \bfseries \textit{Actual Result} &
        \bfseries \textit{Pass/Fail} \\ [0.5ex]
    \hline

    \endfirsthead

    \hline
        \bfseries \textit{Test Code} &
        \bfseries \textit{Test Case} &
        \bfseries \textit{Test Steps} &
        \bfseries \textit{Expected Result} &
        \bfseries \textit{Actual Result} &
        \bfseries \textit{Pass/Fail} \\ [0.5ex]
    \hline
    \endhead % all the lines above this will be repeated on every page
    \hline

    \csvreader[
        late after line=\\,
        before reading={\catcode`\#=12},after reading={\catcode`\#=6}
    ]{tables/hasil/iterations/1/blackbox/engine-speed.csv}
    {1=\code, 2=\case, 3=\step, 4=\expect, 5=\actual, 6=\status}
    {\code & \case & \step & \expect & \actual & \status} \\

    \bottomrule
\end{longtable}

\end{landscape}

\subsection{Iterasi 2}

\subsubsection{Analisis}

Pada iterasi ini, dilanjutkan pengembangan fitur pada halaman Fuel Consumption, Running Hour, dan Data Log. Untuk waktu pengerjaan relatif lebih singkat dibandingkan dengan iterasi pertama karena pada Halaman Fuel Consumption dan Running Hour memiliki algoritma yang identik. Kebutuhan fungsional Fuel Consumption, Running Hour, dan Data Log secara berturut-turut dirincikan pada Gambar \ref{fig:fr-fc}, Gambar \ref{fig:fr-rh}, dan Gambar \ref{fig:fr-dl}.

\begin{figure}[!h]
    \includegraphics[width=.8\linewidth, center]{images/hasil/iterations/2/fr-rh.png}
    \caption{Kebutuhan Fungsional Fuel Consumption}
    \label{fig:fr-fc}
\end{figure}

\begin{figure}[!h]
    \includegraphics[width=.8\linewidth, center]{images/hasil/iterations/2/fr-rh.png}
    \caption{Kebutuhan Fungsional Running Hour}
    \label{fig:fr-rh}
\end{figure}

\newpage

\begin{figure}[!h]
    \includegraphics[width=.8\linewidth, center]{images/hasil/iterations/2/fr-dl.png}
    \caption{Kebutuhan Fungsional Data Log}
    \label{fig:fr-dl}
\end{figure}

\newpage

\subsubsection{Desain}

Berikut merupakan desain wireframe low fidelity pada halaman Fuel Consumption, Running Hour, dan Data Log. Pada halaman Fuel Consumption terdapat informasi singkat mengenai penggunaan bahan bakar tiap mesin dan nilai bahan bakar perkategori operasi FCRV. Lalu terdapat nilai jumlah bahan bakar yang dirata-ratakan dengan interval satu jam dalam rentang waktu satu hari. Pengguna dapat menentukan hari yang diinginkan dengan melakukan filter tanggal yang tersedia di area komponen grafik.

\begin{figure}[!h]
    \includegraphics[width=1.05\linewidth, center]{images/hasil/iterations/2/lofi-fc.png}
    \caption{Wireframe Halaman Fuel Consumption}
    \label{fig:lofi-fc}
\end{figure}

Pada halaman running hour, data akan disajikan dalam format tabel untuk memudahkan pengguna dalam membaca data running hour mesin hari ke hari. Pengguna juga dapat melakukan filter pada rentang tanggal yang diinginkan dengan batas maksimal 30 hari untuk menjaga stabilitas performa dari sistem.

\begin{figure}[!h]
    \includegraphics[width=1.05\linewidth, center]{images/hasil/iterations/2/lofi-rh.png}
    \caption{Wireframe Halaman Running Hour}
    \label{fig:lofi-rh}
\end{figure}

\newpage

Pada Halaman Data Log, data juga disajikan dalam format tabel dengan rentang interval satu menit. Pengguna dapat melakukan filter pada rentang tanggal yang diinginkan dengan batas maksimal 30 hari. Selain dari itu, pengguna juga dapat mengekspor data tersebut dalam format Comma Separated Value (CSV) yang akan dijelaskan lebih lanjut pada Iterasi 5.

\begin{figure}[!h]
    \includegraphics[width=1.05\linewidth, center]{images/hasil/iterations/2/lofi-dl.png}
    \caption{Frontend Halaman Data Log}
    \label{fig:lofi-dl}
\end{figure}

\newpage

\subsubsection{Coding}

Halaman Fuel Consumption merupakan halaman untuk memantau konsumsi bahan bakar berdasarkan kategori operasi FCRV dan histori tren dari bahan bakar yang disajikan dalam bentuk grafik batang. Pengguna dapat melakukan filter tanggal untuk mendapatkan data sesuai dengan batas waktu yang ditentukan.

\begin{figure}[!h]
    \includegraphics[width=1.05\linewidth, center]{images/hasil/iterations/2/fe-fc.png}
    \caption{Frontend Halaman Fuel Consumption}
    \label{fig:fe-fc}
\end{figure}

Halaman ini sepenuhnya menggunakan Client Side Rendering (CSR). Hal
ini memungkinkan pengguna untuk melakukan filter data tanpa perlu memuat ulang
halaman Fuel Consumption sehingga memaksimalkan pengalaman penggunaan sistem.

\newpage

Halaman Running Hour merupakan halaman untuk memantau running hour tiap mesin yang disajikan dalam format tabel. Pengguna dapat melakukan filter rentang tanggal untuk mendapatkan data sesuai dengan batas waktu yang ditentukan. Halaman ini menggunakan Client Side Rendering (CSR) dikarenakan terdapat filter pencarian tanggal pada komponen data table.

\begin{figure}[!h]
    \includegraphics[width=1.05\linewidth, center]{images/hasil/iterations/2/fe-rh.png}
    \caption{Frontend Halaman Running Hour}
    \label{fig:fe-rh}
\end{figure}

Halaman Data Log merupakan halaman untuk mendapatkan data mentah kecepatan mesin yang disajikan dalam format tabel. Pengguna dapat melakukan filter rentang tanggal untuk mendapatkan data sesuai dengan batas waktu yang ditentukan. Serupa dengan Halaman Running Hour, halaman ini juga menggunakan Client Side Rendering (CSR) dikarenakan terdapat filter pencarian tanggal dan waktu pada komponen data table.

\begin{figure}[!h]
    \includegraphics[width=1.05\linewidth, center]{images/hasil/iterations/2/fe-dl.png}
    \caption{Wireframe Halaman Data Log}
    \label{fig:fe-dl}
\end{figure}

\newpage

\subsubsection{Whitebox Testing}

Pada tahap ini, dilakukan pengujian unit test pada halaman Fuel Consumption, Running Hour, dan Data Log. Daftar test case dan hasilnya dapat dilihat pada Gambar ... dan Gambar ... .

\begin{landscape}
    \subsubsection{Blackbox Testing}
    Pada tahap ini, dilakukan pengujian oleh mitra pada halaman Fuel Consumption, Running Hour, dan Data Log.

    \input{tables/hasil/iterations/2/blackbox/fuel-cons.tex}
    \input{tables/hasil/iterations/2/blackbox/running-hour.tex}
    \newpage
    \begin{longtable}[!h]
    {
            p{0.2\textwidth}
            p{0.3\textwidth}
            p{0.3\textwidth}
            p{0.3\textwidth}
            p{0.3\textwidth}
            p{0.15\textwidth}
    }
    \caption{Blackbox Testing Halaman Data Log}
    \label{tab:it1-blackbox-dl} \\

    \hline
        \bfseries \textit{Test Code} &
        \bfseries \textit{Test Case} &
        \bfseries \textit{Test Steps} &
        \bfseries \textit{Expected Result} &
        \bfseries \textit{Actual Result} &
        \bfseries \textit{Pass/Fail} \\ [0.5ex]
    \hline

    \endfirsthead

    \hline
        \bfseries \textit{Test Code} &
        \bfseries \textit{Test Case} &
        \bfseries \textit{Test Steps} &
        \bfseries \textit{Expected Result} &
        \bfseries \textit{Actual Result} &
        \bfseries \textit{Pass/Fail} \\ [0.5ex]
    \hline
    \endhead % all the lines above this will be repeated on every page
    \hline

    \csvreader[
        late after line=\\,
        before reading={\catcode`\#=12},after reading={\catcode`\#=6}
    ]{tables/hasil/iterations/2/blackbox/data-log.csv}
    {1=\code, 2=\case, 3=\step, 4=\expect, 5=\actual, 6=\status}
    {\code & \case & \step & \expect & \actual & \status} \\

    \bottomrule
\end{longtable}

\end{landscape}

\subsection{Iterasi 3}

\subsubsection{Analisis}

Pada iterasi ini, dilanjutkan pengembangan Sistem Admin yang telah disediakan oleh Django Administration melalui model yang dibuat. Kebutuhan fungsional Sistem Admin meliputi FCRV Threshold Config, User Management, dan Vessel Management yang secara berturut-turut dirincikan pada Gambar \ref{fig:fr-fcrv}, Gambar \ref{fig:fr-user}, dan Gambar \ref{fig:fr-vessel}.

\begin{figure}[!h]
    \includegraphics[width=.8\linewidth, center]{images/hasil/iterations/3/fr-fcrv.png}
    \caption{Kebutuhan Fungsional FCRV Threshold Config}
    \label{fig:fr-fcrv}
\end{figure}

\begin{figure}[!h]
    \includegraphics[width=.8\linewidth, center]{images/hasil/iterations/3/fr-user.png}
    \caption{Kebutuhan Fungsional User Management}
    \label{fig:fr-user}
\end{figure}
\begin{figure}[!h]
    \includegraphics[width=.8\linewidth, center]{images/hasil/iterations/3/fr-vessel.png}
    \caption{Kebutuhan Fungsional Vessel Management}
    \label{fig:fr-vessel}
\end{figure}
\begin{figure}[!h]
    \includegraphics[width=.8\linewidth, center]{images/hasil/iterations/3/fr-login.png}
    \caption{Kebutuhan Fungsional Admin Login}
    \label{fig:fr-login-admin}
\end{figure}

\begin{figure}[!h]
    \includegraphics[width=.8\linewidth, center]{images/hasil/iterations/3/fr-fcrv.png}
    \caption{Kebutuhan Fungsional Admin Logout}
    \label{fig:fr-logout-admin}
\end{figure}

\newpage

\subsubsection{Desain}

Pada Sistem Admin, tidak dilakukan desain wireframe dikarenakan fitur bawaan framewok Django yang telah menyediakan Sistem Admin secara otomatis bernama Django administration.

\subsubsection{Coding}

Halaman Admin memungkinkan pengguna untuk melakukan operasi CRUD (create, read, update, dan delete) pada data pengguna, kapal, dan FCRV threshold. Hasil halaman yang dibuat oleh Django administration dapat dilihat pada Gambar ... hingga Gambar ... .

\subsubsection{Whitebox Testing}

Pada Django administration, seluruh komponen web dihasilkan oleh library bawaan dari package restapi yang telah diabstraksi sedemikian rupa dan telah memiliki alur pengujian sendiri. Sehingga, tidak dimungkinkan untuk dilakukan pengujian melalui unit test yang dibuat sendiri.

\begin{landscape}
    \subsubsection{Blackbox Testing}

    \input{tables/hasil/iterations/3/blackbox/user.tex}
    \newpage
    \input{tables/hasil/iterations/3/blackbox/vessel.tex}
    \newpage
    \input{tables/hasil/iterations/3/blackbox/autentikasi.tex}
\end{landscape}

\subsection{Iterasi 4}

\subsubsection{Analisis}

Pada iterasi 4, difokuskan untuk membuat fitur spesifik seperti membuat laporan kecepatan mesin harian dan laporan konsumsi bahan bakar harian yang secara berturut-turut terdapat pada Halaman Engine Speed dan Fuel Consumption. Kebutuhan fungsional dapat dilihat pada Gambar \ref{fig:fr-generate-es-report} dan Gambar \ref{fig:fr-generate-fuel-report}.

\begin{figure}[!h]
    \includegraphics[width=.8\linewidth, center]{images/hasil/iterations/4/fr-generate-es-report.png}
    \caption{Kebutuhan Fungsional Generate Engine Speed Daily Report}
    \label{fig:fr-generate-es-report}
\end{figure}

\begin{figure}[!h]
    \includegraphics[width=.8\linewidth, center]{images/hasil/iterations/4/fr-generate-fuel-report.png}
    \caption{Kebutuhan Fungsional Generate Fuel Consumption Daily Report}
    \label{fig:fr-generate-fuel-report}
\end{figure}

\subsubsection{Desain}

Pada tahap ini, dilakukan desain laporan bahan bakar yang kemudian dapat dibuat secara dinamis oleh sistem. Laporan kecepatan mesin dan bahan bakar dapat dilihat pada Gambar \ref{fig:es-report} dan Gambar \ref{fig:fc-report}.

\begin{figure}[!h]
    \includegraphics[width=1\linewidth, center]{images/hasil/iterations/4/es-report.png}
    \caption{Contoh Laporan Kecepatan Mesin}
    \label{fig:es-report}
\end{figure}

\begin{figure}[!h]
    \includegraphics[width=1\linewidth, center]{images/hasil/iterations/4/fc-report.png}
    \caption{Contoh Laporan Konsumsi Bahan Bakar}
    \label{fig:fc-report}
\end{figure}

\newpage

\subsubsection{Coding}
\subsubsection{Whitebox Testing}

\begin{landscape}
    \subsubsection{Blackbox Testing}

    \input{tables/hasil/iterations/4/blackbox/generate-es-report.tex}
    \newpage
    \input{tables/hasil/iterations/4/blackbox/generate-fuel-report.tex}
\end{landscape}

\subsection{Iterasi 5}

\subsubsection{Analisis}

Pada iterasi terakhir, dilakukan pengembangan fitur ekspor data yang terdapat pada Halaman Data Log, autentikasi sistem, Halaman OP41 Report, dan Halaman Overview. Kebutuhan fungsional dapat dilihat pada Gambar \ref{fig:fr-export-data} hingga Gambar \ref{fig:fr-op41}

\begin{figure}[!h]
    \includegraphics[width=1\linewidth, center]{images/hasil/iterations/5/fr-export-data.png}
    \caption{Kebutuhan Fungsional Ekspor Data}
    \label{fig:fr-export-data}
\end{figure}

\begin{figure}[!h]
    \includegraphics[width=1\linewidth, center]{images/hasil/iterations/5/fr-login-user.png}
    \caption{Kebutuhan Fungsional User Login}
    \label{fig:fr-login-user}
\end{figure}

\begin{figure}[!h]
    \includegraphics[width=1\linewidth, center]{images/hasil/iterations/5/fr-logout-user.png}
    \caption{Kebutuhan Fungsional User Logout}
    \label{fig:fr-logout-user}
\end{figure}

\begin{figure}[!h]
    \includegraphics[width=1\linewidth, center]{images/hasil/iterations/5/fr-op41.png}
    \caption{Kebutuhan Fungsional OP41 Report}
    \label{fig:fr-op41}
\end{figure}

\newpage

\subsubsection{Desain}

Dilakukan desain pada Halaman Login, OP41 Report, dan Overview yang dapat dilihat pada Gambar \ref{fig:lofi-login}, Gambar \ref{fig:lofi-op41} dan Gambar \ref{fig:lofi-overview} secara berturut-turut.

\begin{figure}[!h]
    \includegraphics[width=1.05\linewidth, center]{images/hasil/iterations/5/lofi-login.png}
    \caption{Wireframe Halaman Login}
    \label{fig:lofi-login}
\end{figure}

\begin{figure}[!h]
    \includegraphics[width=1.05\linewidth, center]{images/hasil/iterations/1/lofi-op41.png}
    \caption{Wireframe Halaman OP41 Report}
    \label{fig:lofi-op41}
\end{figure}

\begin{figure}[!h]
    \includegraphics[width=1.05\linewidth, center]{images/hasil/iterations/5/lofi-overview.png}
    \caption{Wireframe Halaman Overview}
    \label{fig:lofi-overview}
\end{figure}

\newpage

\subsubsection{Coding}

Halaman OP41 memuat informasi running hour dan fuel consumption pada tiap kategori operasi berdasarkan Dokumen FCRV yang dapat menjadi acuan awak kapal untuk mengisi laporan. Jika data belum masuk semua maka ditampilkan alert agar awak kapal tidak mengisi laporan sebelum data pada hari tersebut telah tersinkron. Halaman OP41 dapat dilihat pada Gambar \ref{fig:fe-op41}.

\begin{figure}[!h]
    \includegraphics[width=1\linewidth, center]{images/hasil/iterations/5/fe-op41.png}
    \caption{Frontend OP41 Report}
    \label{fig:fe-op41}
\end{figure}

\newpage

\subsubsection{Whitebox Testing}

\begin{landscape}
    \subsubsection{Blackbox Testing}

    \input{tables/hasil/iterations/5/blackbox/export-data.tex}
    \newpage
    \input{tables/hasil/iterations/3/blackbox/autentikasi.tex}

\end{landscape}
    
\chapter{KESIMPULAN DAN SARAN}


\section{Kesimpulan}
Berdasarkan pengerjaan tugas akhir yang telah dilakukan, didapakatkan kesimpulan sebagai berikut.

\begin{enumerate}
    \item Sistem Monitoring telah selesai dikembangkan selama 5 iterasi, dimulai dari pengerjaan fitur utama, sistem admin, dan kemudian fitur pendukung dan tambahan. Tiap iterasi memiliki rentang waktu pengerjaan maksimum 1 pekan yang diurutan berdasarkan prioritas dan waktu pengerjaan.
    \item Seluruh \textit{task} pada implementasi telah diuji melalui dua metode pengujian, yaitu \textit{Whitebox} dan \textit{Blackbox}. \textit{Whitebox} dilakukan dengan membuat test case pada unit test dan dijalankan sebelum sistem diunggah ke \textit{codebase} dan \textit{Blackbox} dilakukan setelah sistem terunggah ke \textit{codebase} dan masuk ke lingkungan pengembang (\textit{dev mode}).
    \item Metode \textit{Extreme Programming} terbukti telah membantu dalam menghadapi perubahan selama pengembangan seperti perubahan pada Iterasi 2, penambahan sistem admin pada Iterasi 3, penambahan halaman Overview, dan penambahan halaman OP41 Report dan Home pada Iterasi 5 tanpa menginterupsi alur dan jadwal pengembangan.
\end{enumerate}

\section{Saran}

Adapun saran untuk penelitian kedepan pada topik ini adalah sebagai berikut.

\begin{enumerate}
    \item Untuk pengembangan kedepan dapat dilanjut dengan memfokuskan pada keselamatan serta pemantauan emisi pada kapal dengan memasang sensor yang sesuai ataupun dapat memanfaatkan variabel-variabel yang ada.
    \item Metode \textit{Extreme Programming} memerlukan disiplin dan komitmen yang tinggi dalam implementasinya, sehingga pengerjaan lebih dari 1 orang direkomendasikan. Hal ini juga sejalan dengan salah satu praktik metode tersebut, yakni \textit{Pair Programming}.
\end{enumerate}
    %-----------------------------------------------------------
    % End Daftar masukan untuk Bab
    %===========================================================

    %===========================================================
    % Daftar Pustaka
    %-----------------------------------------------------------
    \addcontentsline{toc}{chapter}{DAFTAR PUSTAKA}
    \printbibliography[title={DAFTAR PUSTAKA}]
    %-----------------------------------------------------------
    % End Daftar Pustaka
    %===========================================================

    %===========================================================
    % Daftar Lampiran
    %-----------------------------------------------------------
    \appendix
    \addcontentsline{toc}{chapter}{LAMPIRAN}
    \addtocontents{toc}{\protect\setcounter{tocdepth}{0}}
    \chapter*{LAMPIRAN}

\section{Dokumentasi}

\subsection{Wawancara}
\subsection{Lapangan}


\section{\textit{Source Code}}

\begin{enumerate}
    \item \href{https://github.com}{Frontend Halaman Engine Speed}
    \item \href{https://github.com}{Frontend Halaman Overview}
    \item \href{https://github.com}{Frontend Halaman Fuel Consumption}
    \item \href{https://github.com}{Frontend Halaman OP41 Report}
    \item \href{https://github.com}{Frontend Halaman Running Hour}
    \item \href{https://github.com}{Frontend Halaman Data Log}
    \item \href{https://github.com}{Backend Overview}
    \item \href{https://github.com}{Backend Engine Speed}
    \item \href{https://github.com}{Backend Halaman Fuel Consumption}
    \item \href{https://github.com}{Backend Halaman OP41 Report}
    \item \href{https://github.com}{Backend Halaman Running Hour}
    \item \href{https://github.com}{Backend Halaman Data Log}
\end{enumerate}
    %-----------------------------------------------------------
    % End Daftar Lampiran
    %===========================================================
\end{document}
