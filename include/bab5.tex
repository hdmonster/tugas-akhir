
\chapter{KESIMPULAN DAN SARAN}


\section{Kesimpulan}
% \subsection{Citra Digital}
Berdasarkan pengerjaan tugas akhir yang telah dilakukan, didapakatkan kesimpulan sebagai berikut.

\begin{enumerate}
    \item Sistem Monitoring dikembangkan selama 5 iterasi, dimulai dari pengerjaan fitur utama, sistem admin, dan kemudian fitur pendukung dan tambahan. Tiap iterasi memiliki rentang waktu pengerjaan maksimum 1 pekan dengan beban maksimum \textit{story point} sebesar x.
    \item Seluruh task pada implementasi telah diuji melalui dua metode testing, yaitu Whitebox dan Blackbox. Whitebox dilakukan dengan membuat test case pada unit test dan dijalankan sebelum sistem diunggah ke \textit{codebase} dan Blackbox dilakukan setelah sistem terunggah ke \textit{codebase} dan masuk ke lingkungan pengembang (\textit{dev mode}).
    \item Metode Extreme Programming terbukti telah membantu dalam menghadapi perubahan selama pengembangan seperti perubahan pada Iterasi 2, penambahan sistem admin pada Iterasi 3, dan penambahan halaman OP41 Report dan Overview pada Iterasi 5 tanpa mengintervensi alur dan jadwal pengembangan.
\end{enumerate}

\section{Saran}
\blindtext