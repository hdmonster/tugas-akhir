
\chapter{KESIMPULAN DAN SARAN}


\section{Kesimpulan}
Berdasarkan pengerjaan tugas akhir yang telah dilakukan, didapakatkan kesimpulan sebagai berikut.

\begin{enumerate}
    \item Sistem Monitoring telah selesai dikembangkan selama 5 iterasi, dimulai dari pengerjaan fitur utama, sistem admin, dan kemudian fitur pendukung dan tambahan. Tiap iterasi memiliki rentang waktu pengerjaan maksimum 1 pekan yang diurutan berdasarkan prioritas dan waktu pengerjaan.
    \item Seluruh \textit{task} pada implementasi telah diuji melalui dua metode pengujian, yaitu \textit{Whitebox} dan \textit{Blackbox}. \textit{Whitebox} dilakukan dengan membuat test case pada unit test dan dijalankan sebelum sistem diunggah ke \textit{codebase} dan \textit{Blackbox} dilakukan setelah sistem terunggah ke \textit{codebase} dan masuk ke lingkungan pengembang (\textit{dev mode}).
    \item Metode \textit{Extreme Programming} terbukti telah membantu dalam menghadapi perubahan selama pengembangan seperti perubahan pada Iterasi 2, penambahan sistem admin pada Iterasi 3, penambahan halaman Overview, dan penambahan halaman OP41 Report dan Home pada Iterasi 5 tanpa menginterupsi alur dan jadwal pengembangan.
\end{enumerate}

\section{Saran}

Adapun saran untuk penelitian kedepan pada topik ini adalah sebagai berikut.

\begin{enumerate}
    \item Untuk pengembangan kedepan dapat dilanjut dengan memfokuskan pada keselamatan serta pemantauan emisi pada kapal dengan memasang sensor yang sesuai ataupun dapat memanfaatkan variabel-variabel yang ada.
    \item Metode \textit{Extreme Programming} memerlukan disiplin dan komitmen yang tinggi dalam implementasinya, sehingga pengerjaan lebih dari 1 orang direkomendasikan. Hal ini juga sejalan dengan salah satu praktik metode tersebut, yakni \textit{Pair Programming}.
\end{enumerate}