\chapter*{KATA PENGANTAR}

Puji syukur kepada Allah SWT atas berkat dan rahmat-Nya penulis dapat menyelesaikan laporan tugas akhir yang berjudul:

\begin{center}
    "PENGEMBANGAN SISTEM MONITORING MESIN DIESEL BERBASIS IOT PADA PT BISMA JAYA MENGGUNAKAN METODE EXTREME PROGRAMMING (XP)"
\end{center}

Laporan tugas akhir ini merupakan salah satu syarat yang harus
ditempuh untuk menyelesaikan Program Sarjana di Program Studi Sistem
Informasi, Jurusan Matematika dan Teknologi Informasi, Institut Teknologi
Kalimantan (ITK) Balikpapan. Besar terima kasih penulis ucapkan kepada:

\begin{enumerate}[topsep=0pt,itemsep=0pt,partopsep=0pt, parsep=0pt]

    \item Bapak Aidil Kirsan Saputra, S.Kom., M.Tr.Kom, selaku Dosen Pembimbing Utama dan
    Bapak Henokh Lugo Hariyanto, S.Si., M.Sc., selaku Dosen Pembimbing
    Pendamping.

    \item Ibu Sri Rahayu Natasia, S.Komp, M.Si., M.Sc., selaku Koordinator Program
    Studi Sistem Informasi Jurusan Matematika dan Teknologi Informasi ITK.

    \item Bapak/Ibu Dosen dan Bapak/Ibu Tendik Program Studi Sistem Informasi
    Jurusan Matematika dan Teknologi Informasi ITK.

    \item Serta semua pihak yang terlibat dalam penyusunan proposal tugas akhir ini.
\end{enumerate}

Penulis menyadari bahwa laporan tugas akhir ini masih jauh dari sempurna dikarenakan terbatasnya pengalaman dan pengetahuan yang dimiliki penulis. Oleh karena itu, penulis mengharapkan segala bentuk saran serta masukan bahkan kritik yang membangun dari berbagai pihak. Semoga tulisan ini memberikan manfaat kepada semua pihak yang membutuhkan dan terutama untuk penulis.

\vspace{0.5cm}
\begin{flushright}
    Balikpapan, 16 Januari 2024\\
    \vspace{2cm}
    {Haidar Dzaky Sumpena}\\
    NIM {10201043}
\end{flushright}

