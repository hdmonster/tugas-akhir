\chapter*{PENGEMBANGAN SISTEM MONITORING MESIN DIESEL BERBASIS IOT PADA PT BISMA JAYA MENGGUNAKAN METODE EXTREME PROGRAMMING (XP)}
\begin{table}[h]
    \begin{tabular}
        {
            p{.4\textwidth}
            p{.6\textwidth}
            }
            \\
            Nama & : Haidar Dzaky Sumpena
            \\
            NIM & : 10201043
            \\
            Dosen Pembimbing Utama &
            : Aidil Kirsan Saputra, S.Kom., M.Tr.Kom
            \\
            Dosen Pembimbing Pendamping &
            : Henokh Lugo Hariyanto, M.Sc.
        \end{tabular}
    \end{table}

\begin{center}
    \textbf{ABSTRAK}
\end{center}

Dalam penelitian ini telah dimanfaatkan IoT dalam pemantauan jumlah bahan bakar di industri maritim yang dipasang pada kapal milik PT Bisma Jaya. Pembangunan sistem IoT ini didasarkan oleh metode \textit{Extreme Programming} yang menekankan kolaborasi serta pengembangan dinamis. Dipilihnya metode ini juga terbukti mampu menangani umpan balik dan menjaga kualitas kode selama pengembangan sistem berlangsung. Luaran dari penelitian ini adalah Sistem Monitoring berbasis web yang menggunakan teknologi Django dan NextJS. Hasilnya, telah dibangun sistem pemantauan yang dipasang pada 3 kapal dengan interval pengambilan data 1 menit secara konstan. Pengembangan lebih lanjut akan difokuskan pada keselamatan kapal dan pemantauan emisi.


\begin{table}[h]
    \begin{tabular}{ p{0.17\textwidth} p{0.8\textwidth} }
        \\
        \textbf{Kata Kunci :} & \textit{Internet of Things}, sistem \textit{monitoring}, bahan bakar, \textit{Extreme Programming}
    \end{tabular}
\end{table}
