\chapter*{PENGEMBANGAN SISTEM MONITORING MESIN DIESEL BERBASIS IOT PADA PT BISMA JAYA MENGGUNAKAN METODE EXTREME PROGRAMMING (XP)}
\begin{table}[h]
    \begin{tabular}
        {
            p{.4\textwidth}
            p{.6\textwidth}
            }
            \\
            Nama & : Haidar Dzaky Sumpena
            \\
            NIM & : 10201043
            \\
            Dosen Pembimbing Utama &
            : Aidil Kirsan Saputra, S.Kom., M.Tr.Kom
            \\
            Dosen Pembimbing Pendamping &
            : Henokh Lugo Hariyanto, M.Sc.
        \end{tabular}
    \end{table}

\begin{center}
    \textbf{ABSTRAK}
\end{center}

Internet of Things merupakan teknologi yang dapat merevolusi industri melalui kontrol dan pemantauan secara jarak jauh. Teknologi ini dapat diterapkan di berbagai industri termasuk pada transportasi.Tantangan di industri ini adalah memastikan jumlah bahan bakar yang dilaporkan sesuai dengan nilai aktual yang dihabiskan. Sistem monitoring berbasis IoT yang akan dikembangkan dalam penelitian ini berfungsi sebagai penghubung mitra dengan armada yang beroperasi. Metode pengembangan perangkat lunak yang digunakan adalah Extreme Programming yang menekankan kolaborasi serta pengembangan yang dinamis. Diharapkan, dengan diterapkannya sistem monitoring berbasis IoT ini, mitra dapat melakukan pemantauan bahan bakar serta menjaga efektivitas armada yang beroperasi secara real time.


\begin{table}[h]
    \begin{tabular}{ p{0.17\textwidth} p{0.8\textwidth} }
        \\
        \textbf{Kata Kunci :} & \textit{Internet of Things}, sistem \textit{monitoring}, bahan bakar, efisiensi, \textit{extreme programming}
    \end{tabular}
\end{table}
